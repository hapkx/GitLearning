\documentclass{article}

\usepackage{palatino}

\usepackage[UTF8, nocap]{ctex} %如果想使用中文输入的话,可以增加该宏包
\usepackage{amsmath}   %数学公式
\usepackage{xfrac}     %行间分式
\usepackage{graphicx}  %图片

\usepackage{ctex}        %表格
\usepackage{array}
\usepackage{geometry}
\usepackage{booktabs}
\usepackage{tabularx}
\usepackage{fancyhdr}    %页眉页脚

\usepackage[justification=centering]{caption}  %图表标题居中
\usepackage{multirow}
\usepackage{multicol}
\usepackage{arydshln}
\usepackage{booktabs}

\usepackage{cite}   %引用

\newcolumntype{L}[1]{>{\raggedright\arraybackslash}p{#1}}
\newcolumntype{C}[1]{>{\centering\arraybackslash}p{#1}}
\newcolumntype{R}[1]{>{\raggedleft\arraybackslash}p{#1}}
\setlength{\parindent}{0pt}

\usepackage{setspace}
\onehalfspacing  %行间距是字号的1.5倍

\addtolength{\parskip}{.4em}   %增加段间距 0.4em

\title{你好,world!}
\author{xiaopian}
\date{2020.1.1}

\pagestyle{fancy}  %页眉页脚
\lhead{hiahiahia} 
\chead{} 
\rhead{The performance of new graduates} 
\lfoot{From: K. Grant} 
\cfoot{To: Dean A. Smith}
\rfoot{\thepage} 
\renewcommand{\headrulewidth}{0.4pt} 
\renewcommand{\footrulewidth}{0.4pt}

\geometry{a4paper, left = 3.17cm, right = 3.17cm, top = 2.54cm, bottom = 2.54cm}

\begin{document}

\bibliographystyle{unsrt}

\thispagestyle{empty}   %第一页目录页不加页眉页脚
\tableofcontents
\maketitle
你好,world! 

I love you!        %换行要两次回车
\section{你好中国}   %一级标题
中国在East Asia.\\
\textup{中国在East Asia.}\\
意大利斜体:\textit{中国在East Asia.}\\
slanted斜体:\textsl{中国在East Asia.}\\
显示小体大写文本:\textsc{中国在East Asia.}\\
加粗命令:\textbf{中国在East Asia.}
\subsection{Hello Beijing}   %二级标题
北京是capital of China.
\subsubsection{Hello Dongcheng District}  %三级标题
\paragraph{Tian'anmen Square}
is in the center of Beijing
\subparagraph{Chairman Mao}    %subparagraph比paragraph缩近了一点
is in the center of 天安门广场。
\subsection{Hello 山东}
\paragraph{山东大学} is one of the best university in 山东。

\section{FORMULA}
公式啥的
\subsection{formula}
Einstein 's $E=mc^2$   %行内
\[ E=mc^2. \]     %行间,没有编号
\begin{equation*}  %行间,没有编号
    E=mc^2
\end{equation*}

\begin{equation}  %行间,有编号
E=mc^2
\end{equation}
%行内公式的标点,应该放在数学模式的限定符之外,而行间公式则应该放在数学模式限定符之内。
\[ z = r_i\cdot e^{2\pi i}. \]  %上标^,下标_

\paragraph{分式和根式}分式和根式:
$\sqrt{x}$, $\dfrac{1}{2}$,$\frac{1}{2}$. %dfrac:强制行内分数显示为行间的大小
$\sfrac{1}{2}$   %写成1/2的样子(xfrac宏包提供)

\[ \cfrac{a}{1+\cfrac{b}{1+\cfrac{c}{d}}} \]   %cfrac繁分式(但是好像写出来差不多?)
\[ \frac{a}{1+\cfrac{b}{1+\cfrac{c}{d}}} \]

\[ \sqrt{x}, \]    %根式\sqrt{·},分式\frac{分子}{分母}

\[ \frac{1}{2}. \]
\[ \tfrac{1}{2}. \]    %强制行间显示为行内大小 (感觉用处一般般)

\paragraph{符号:}符号:
\[ \pm\; \times \; \div\; \cdot\; \cap\; \cup\;
\geq\; \leq\; \neq\; \approx \; \equiv \]
\[ \sum, \prod, \lim, \int \]

$ \sum_{i=1}^n i\quad \prod_{i=1}^n $  %求和 求积


limit限制行高$ \sum\limits _{i=1}^n i\quad \prod\limits _{i=1}^n $

\[ \lim_{x\to0}x^2 \quad \int_a^b x^2 dx \]  %lim和积分
\[ \lim\nolimits _{x\to0}x^2\quad \int\nolimits_a^b x^2 dx \]

\paragraph{定界符}定界符:
$(), [], \{\}, \langle\rangle $
$ \lvert\rvert, \lVert\rVert, \|$
\[ \Biggl(\biggl(\Bigl(\bigl((x)\bigr)\Bigr)\biggr)\Biggr) \]
\[ \Biggl[\biggl[\Bigl[\bigl[[x]\bigr]\Bigr]\biggr]\Biggr] \]
\[ \Biggl \{\biggl \{\Bigl \{\bigl \{\{x\}\bigr \}\Bigr \}\biggr \}\Biggr\} \]
\[ \Biggl\langle\biggl\langle\Bigl\langle\bigl\langle\langle x
\rangle\bigr\rangle\Bigr\rangle\biggr\rangle\Biggr\rangle \]
\[ \Biggl\lvert\biggl\lvert\Bigl\lvert\bigl\lvert\lvert x
\rvert\bigr\rvert\Bigr\rvert\biggr\rvert\Biggr\rvert \]
\[ \Biggl\lVert\biggl\lVert\Bigl\lVert\bigl\lVert\lVert x
\rVert\bigr\rVert\Bigr\rVert\biggr\rVert\Biggr\rVert \]

\[ x_1,x_2,\dots ,x_n\quad 1,2,\cdots ,n\quad  %dots cdots位置不同
\vdots\quad \ddots \]

\paragraph{矩阵}
\[ \begin{pmatrix} a&b\\c&d \end{pmatrix} \quad  %圆括号
\begin{bmatrix} a&b\\c&d \end{bmatrix} \quad     %方括号
\begin{Bmatrix} a&b\\c&d \end{Bmatrix} \quad     %花括号
\begin{vmatrix} a&b\\c&d \end{vmatrix} \quad     %单竖线
\begin{Vmatrix} a&b\\c&d \end{Vmatrix} \]        %双竖线
Marry has a little matrix $ ( \begin{smallmatrix} a&b\\c&d \end{smallmatrix} ) $.   %行内矩阵(好可爱啊)

\paragraph{长公式}长公式:
\begin{multline}
    x = a+b+c+{} \\
    d+e+f+g
\end{multline}  %multline*是没有编号

\[\begin{aligned}
    x ={}& a+b+c+{} \\
    &d+e+f+g
    \end{aligned} \]

公式组:
\begin{gather}
    a = b+c+d \\
    x = y+z
    \end{gather}
    \begin{align}
    a &= b+c+d \\
    x &= y+z
    \end{align}

分段函数:
\[ y= \begin{cases}
    -x,\quad x\leq 0 \\
    x,\quad 0<x<2 \\
    x^2,\quad x\ge 2
    \end{cases} \]

\section{图表}
\subsection{插入图片}

\includegraphics{aaa.jpg}\\
\includegraphics[width = .5\textwidth]{aaa.jpg}

\begin{figure}[htbp]
    \centering
    \includegraphics{aaa.jpg}
    \caption{有图有真相}
    \label{fig:myphoto}
\end{figure}

\subsection{表格}
\begin{tabular}{lc|r|}  %第一列左对齐 第二列居中,第三列右对齐
    \hline
   操作系统& 发行版& 编辑器\\
    \hline
   Windows & MikTeX & TexMakerX \\
    \hline
   Unix/Linux & teTeX & Kile \\
    \hline
   Mac OS & MacTeX & TeXShop \\
    \hline
   通用& TeX Live & TeXworks \\
    \hline
   \end{tabular}\\

\renewcommand\arraystretch{2}  %宽松的表格
宽松的表格\\ \\
\begin{tabular}{|lc|r|}  
    \hline
   操作系统& 发行版& 编辑器\\
    \hline
   Windows & MikTeX & TexMakerX \\
    \hline
   Unix/Linux & teTeX & Kile \\
    \hline
   Mac OS & MacTeX & TeXShop \\
    \hline
   通用& TeX Live & TeXworks \\
    \hline
   \end{tabular}
 \\ \\ \\

$\left(\begin{tabular}{ccc|c}
        a11 & a12 & a13  & b1  \\
        a21 & a22  & a23 & b2  \\ 
        a31 & a32  & a33 & b3  \\     
\end{tabular}\right)$

\begin{center}
    \begin{tabular}{|c|c|c|}
    \hline
    \multicolumn{2}{|c|}{成绩}  \\    %合并两列
    \hline
    语文  &   数学  \\   
    \hline    
    87    & 100  \\
    \hline  
    \end{tabular}

    \begin{tabular}{|c|r|r|}
        \hline
        & \multicolumn{2}{|c|}{成绩}   \\ \cline{2-3}
        姓名   & 语文 & 数学 \\ 
        \hline
        张三   & 87 & 100 \\
        \hline
    \end{tabular}
\end{center}

\paragraph{三线表} 绘制三线表

    \begin{table}[h]   %[htbp],意思就是优先放在此处,其次是每页的顶端,再次是底端
    \centering
    \caption{三线表}
    \begin{tabular}{clllll}
     \toprule
     \multirow{2}{*}{obfuscations} & \multicolumn{5}{l}{Transformations}   \\
     \cline{2-6}    % 画一条线
     & Renaming & Dead code removal & control flow obfuscation & string encryption & code encryption \\
     \midrule
     1 & √ & ×  & × & √  & × \\
     2 & √ & ×  & × & √  & × \\
     3 & √ & ×  & × & √  & × \\
     4 & √ & ×  & × & √  & × \\
     \bottomrule
    \end{tabular}
    \end{table}


   \paragraph{指定宽度的表格}
   \begin{table}[h]
   \caption{指定宽度表格\cite{doi:10.1080/07388550500346359}} 
   \begin{tabular}{R{1cm}|C{3cm}|C{10cm}}
   \bottomrule[2pt]
   年份 & 获奖者 & 获奖原因\\
   \specialrule{2pt}{0pt}{0pt}   %第1个参数是线条的粗细,第2个是上方的间隙、第3个是下方的间隙
   1966 & Alan J. Perlis & 先进编程技术和编译架构方面的贡献\\
   \hline
   1971 & John McCarthy & Lisp语言、程序语义、程序理论、人工智能方面的贡献\\
   \hline
   1972 & E. W. Dijkstra & 对开发Algol做出了原理性贡献\\
   \hline
   1977 & John Backus & 在高级语言方面所做出的具有广泛和深远意义的贡献,特别是在Fortran语言方面\\
   \toprule[2pt]
   \end{tabular}
   \end{table}

   \begin{table}[h]
    \caption{指定宽度表格\cite{jambeck2015plastic}} 
    \begin{tabular}{R{1cm}|C{3cm}|C{10cm}}
    \bottomrule[2pt]
    年份 & 获奖者 & 获奖原因\\
    \specialrule{2pt}{0pt}{0pt}   %第1个参数是线条的粗细,第2个是上方的间隙、第3个是下方的间隙
    1966 & Alan J. Perlis & 先进编程技术和编译架构方面的贡献\\
    \hline
    1971 & John McCarthy & Lisp语言、程序语义、程序理论、人工智能方面的贡献\\
    \hline
    1972 & E. W. Dijkstra & 对开发Algol做出了原理性贡献\\
    \toprule[2pt]
    \end{tabular}
    \end{table}

\section{PAPER NOTES}
\subsection{结构}
\begin{itemize}
    \item 跟上一篇差不太多。\cite{1}   \\   
    交代背景与现状,简述问题,给出assumptions和notations,建立PWH Model,
          分别对P W H三个方面进行初步模拟预测。完善PWH Model。建立IPWH Model,引入alternatives、
          policy intervention、eco-awareness、recovery的影响因素。
          这一趴不仅仅给出预测结果,而且用来设立目标。两个model的结果需要进行比较,来说明IPWH下
          目标的可达成性。然后说明达成目标后的影响,注意多方面,且量化。
          后一段是全球不同地区的差异性。根据评分分级(这里的几张图感觉都很漂亮)。
          最后就是sensity analysis和优劣分析,和总结。(sensity analysis那张图真好看)
    \item 中间有两张流程图,感觉可以借鉴下,比如这篇参数比较多、过程相对有一丢丢复杂的情况。
        (也比较有助于让我自己看懂orz)
    \item 最后有Memo和Appendice。Memo里的那张图好好看啊。\\
    \begin{figure}[htbp]
        \centering
        \includegraphics{Memo.jpg}
        \caption{Memo}
        \label{fig:myphoto}
    \end{figure}
    \item 这篇中间的公式推导建模说实话我有点没太看明白。。。有些数据的来源感觉好像并没有交代的很清楚?比如β$_1$,α$_0$,α$_1$,还有个地方直接here we choose 10\%.
    以及policy intervention、eco-awareness是怎么量化的,感觉用指数评估有点简单了?
\end{itemize}

\subsection{\LaTeX 方面}
正在熟悉最基本的操作,用的还不太熟练。\\
这一篇大部分还相对比较友好,图片、表格啥的比较常规。不过也有很多难点。
\subparagraph{1.}变量、符号很多,希腊字母很多,记不住需要查表。
\subparagraph{2.}公式比较复杂,一层一层嵌套,试了一下感觉有点晕(大概率还是操作不熟练的原因,要多试验练习)
\subparagraph{3.}Memo和Appendice的排版还没有琢磨透,reference的用的也有点磕绊。\\
                总的来说就是熟练度还远远远远不够。可以一边琢磨模板排版的同时一边熟悉操作,一边再找一找其他的排版样式。\\
                


\begin{thebibliography}{99}
    \bibitem{1} D.~E. KNUTH   The \TeX{}book  the American
    Mathematical Society and Addison-Wesley
    Publishing Company , 1984-1986.
    \bibitem{2}Lamport, Leslie,  \LaTeX{}: `` A Document Preparation System '',
    Addison-Wesley Publishing Company, 1986.
\end{thebibliography}

\end{document}